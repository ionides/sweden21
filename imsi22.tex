%\documentclass[handout]{beamer}
\documentclass{beamer}

\usepackage{amsmath,amsfonts,amssymb,graphicx} 
\input{header.tex}
\input{header-ms.tex}
\input{theorems1.tex}

\begin{document}

\begin{frame}
  
\begin{center}
  {\Large\bf Inference for metapopulation dynamics}

%% ABSTRACT: Microbial populations can demonstrate highly nonlinear, partially observable, stochastic dynamics. Parameter estimation and model criticism are inferential challenges which are tractable using modern methods applicable to general low-dimensional ecological models. Collections of interacting ecosystems at different locations raise additional challenges associated with high-dimensional Monte Carlo inference. Questions arise about when it is necessary to develop models for dynamic coupling between metapopulations, and if so how to carry out inference in this setting. We seek general-purpose methods that allow scientists to ask and answer questions for metapopulation dynamics in the context of scientifically useful models. We present recent progress and ongoing developments.


\vspace{2mm}

Edward Ionides\\
University of Michigan, Department of Statistics

\vspace{8mm}

Multiscale Microbial Communities\\
Dynamical Models, Ecology, and One Health\\
February 22, 2022


\hspace{3mm}

Slides are at \url{https://ionides.github.io/talks/imsi22.pdf}
%% 8:15am EST

\vspace{8mm}

Joint work with\\
Kidus Asfaw, Ning Ning, Joonha Park and Aaron King

\end{center}

\end{frame}


\newcommand\challengeSep{\vspace{3mm}}

\begin{frame}{Inference challenges in population dynamics}

  \begin{enumerate}
\item Combining measurement noise and process noise.
\item Including covariates in mechanistically plausible ways.
\item  Continuous time models.
\item  Modeling and estimating interactions in coupled systems.
\item  Dealing with unobserved variables.
\item  \myemph{Modeling spatial-temporal dynamics}.
\item  \myemph{Studying population dynamics via genetic sequence data}.
  \end{enumerate}

  \vspace{3mm}
  
  1--6 are from Bjornstad \& Grenfell ({\it Science}, 2001).\\
  7 is from Grenfell et al ({\it Science}, 2004).\\
  1--5 are largely solved, from a methodological perspective.
  
  
\end{frame}

\begin{frame}{Example: Pre-vaccination measles in England \& Wales}

\vspace{-3mm}

\begin{center}
\includegraphics[width=8.5cm]{he10-data.pdf}


\end{center}

\vspace{-2mm}
  
\end{frame}

\begin{frame}{Time series data, panel data \& spatiotemporal data}

  \bi
\item Looking at one unit (town) is \myemph{time series analysis}.

  \item Joint modeling of a few units (say, 2 or 3) is \myemph{multivariate time series analysis}.

\item Analysis of many time series, without consideration of dynamic interactions, is \myemph{panel data analysis}.

\item Allowing for coupling between units, we get \myemph{spatiotemporal analysis}, which in our context is \myemph{metapopulation analysis}.

  \ei
  
Question: When should we avoid inference for spatiotemporal models? When do we need to consider coupling? How?

\end{frame}

\begin{frame}{Desiderata}

  \begin{itemize}
    \item We want to be able to fit arbitrary dynamic models. The limitations should be our scientific creativity and the information in the data.

    \item In practice, that means using \myemph{plug-and-play} methods which need a simulator from the model but not nice parametric equations.

    \item We want statistically efficient inference, to extract all the information in the data.

    \item In practice, that means using likelihood-based methods.

      \item  In the time series case, iterated particle filtering (IF2) implemented in the R package \texttt{pomp} enables Masters-level statisticians to do this (\url{https://ionides.github.io/531w22/}). The science may be hard, but the statistics is becoming routine.
      \end{itemize}
  \end{frame}



\begin{frame}{Panel data}

\bi
\item To investigate epidemiological dynamics in multiple cities, one can consider each city independently, perhaps modeling a background immigration rate of infections for each city.

\item \myemph{Decoupling} leads to panel data analysis, by assumption. Iterated filtering methods extend to panel data (Breto et al, {\it Journal of the American Statistical Association}, 2019).

\item We must decide which parameters should be modeled as \myemph{shared} vs \myemph{unit-specific}.

\item The consequences of decoupling are becoming easier to study with the development of statistical inference methods for coupled systems, i.e., metapopulation dynamics.

  \ei

  \end{frame}

\begin{frame}{The curse of dimensionality}

  \bi
  \item
    Particle filter (PF) methods are effective for inference on low-dimensional nonlinear partially observed stochastic dynamic systems. They scale exponentially badly.

\item Extending the successes of particle filter methods from time series data to metapopulation data is becoming possible.

\item Algorithms under consideration:\\
  {\bf
  Bagged filters (BF, IBF)\\
  Ensemble Kalman filter (EnKF, IEnKF)\\
  Guided intermiediate resampling filter (GIRF, IGIRF)\\
  Block particle filter (BPF, IBPF)\\
  }
  
\item Filters estimate latent states and evaluate the likelihood.
\item Each filter has an iterated version which estimates parameters by repeated filtering using stochastic parameter perturbations.

\item These algorithms are all implemented in an R package, \texttt{spatPomp}.
  
  \ei
  
\end{frame}

\begin{frame}{Filtering $U$ units of a coupled measles SEIR model}

\vspace{-3mm}

\begin{center}
\includegraphics[width=10cm]{mscale_loglik_plot-1.pdf}


\end{center}

\vspace{-2mm}

Simulated data using a gravity model with geography, demography and transmssion parameters corresponding to UK pre-vaccination measles (Ionides et al, {JASA}, 2021).

%\end{center}

\end{frame}

\begin{frame}
\frametitle{$U=40$ units for a coupled measles SEIR model}

\vspace{-2.7mm}

\begin{center}
\includegraphics[width=8cm]{slice_image_plot-1.pdf}
\end{center}

\vspace{-3mm}

{\bf A}. Simulated Susceptible-Exposed-Infected-Recovered dynamics coupled with a gravity model (log of biweekly reported cases).

{\bf B}. Measles UK pre-vaccination case reports for the 40 largest cities.




%\end{center}

\end{frame}


\begin{frame}{More on the block particle filter}

\bi
\item BPF worked quickly, easily and reliably on our measles model filtering experiments.

\item This motivated us to develop an IBPF for parameter estimation.
   
\item BPF has theoretical support in some situations (Rebeschini \& Van Handel, {\it Annals of Applied Probability}, 2015).

\item BPF was independently proposed as the ``factored particle filter'' by Ng et al (2002, {\it Proc. 18th Conference on Uncertainty and Artificial Intelligence}) but not widely popularized.

\ei

\end{frame}

\begin{frame}{Particle filter (PF)}

  \begin{columns}
    \begin{column}{0.48\linewidth}
      \begin{center}
      {\bf \textcolor{blue}{Evolutionary analogy}}

      \vspace{5mm}
      
      {\bf Mutation}

      $\downarrow$

      {\bf Fitness}

      $\downarrow$

      {\bf Natural selection}
      
      \end{center}
    \end{column}
     \begin{column}{0.48\linewidth}
      \begin{center}
      {\bf \textcolor{blue}{Particle filter algorithm}}

      \vspace{5mm}
      
      {\bf Predict: stochastic dynamics}

      $\downarrow$

      {\bf Measurement: weight}

      $\downarrow$

      {\bf Filter: resample}
      \end{center}
    \end{column}
  \end{columns}

  \vspace{15mm}
  
    \begin{myitemize}
  \item PF is an evolutionary algorithm with good mathematical properties: an unbiased likelihood estimate and consistent latent state distribution.
  \end{myitemize}

\end{frame}
  
\begin{frame}{Block particle filter (BPF)}

  \begin{columns}
    \begin{column}{0.48\linewidth}
      \begin{center}
      {\bf \textcolor{blue}{Evolutionary analogy}}

      \vspace{5mm}
      
      {\bf Mutation}

      $\downarrow$

      {\bf Fitness\\
      for each chromosome}

      $\downarrow$

      {\bf Natural selection\\
      for each chromosome}

      $\downarrow$

      {\bf Recombine chromosomes}
      
      \end{center}
    \end{column}
     \begin{column}{0.48\linewidth}
      \begin{center}
      {\bf \textcolor{blue}{Block particle filter}}

      \vspace{5mm}
      
      {\bf Predict: stochastic dynamics}

      $\downarrow$

      {\bf Measurement: weight\\
      for each block}

      $\downarrow$

      {\bf Filter: resample\\
      for each block}

      $\downarrow$

      {\bf Recombine blocks}
      \end{center}
    \end{column}
  \end{columns}

  \vspace{5mm}
  
    \begin{myitemize}
    \item Blocks in BPF allow recombination (reassortment of chromosomes in sexual reproduction) in the evolutionary analogy.
    \item Blocks are a partition of the metapopulation units. Our experiments suggest treating each sub-population (i.e., city) as a block.

  \end{myitemize}

\end{frame}

\begin{frame}{An iterated block particle filter for parameter estimation}


  \begin{center}
    
  \includegraphics[trim={0 0 0 10mm},clip,width=9cm]{IBPF_workflow.pdf}


  \end{center}
  
\end{frame}

\begin{frame}{Auto-regression of spatial perturbations for shared parameters}

\vspace{-10mm}

\begin{center}
\includegraphics[width=8.5cm]{ibpf/boxplot-spat-reg.pdf}

\end{center}

\vspace{-2mm}
  
\end{frame}

\begin{frame}{Random perturbations must be smaller to match larger number  $(20\times 13)$ of parameters}
    
\vspace{-10mm}

\begin{center}
\includegraphics[width=8.5cm]{ibpf/boxplot-rw-sd.pdf}

\end{center}

\vspace{-2mm}
  
\end{frame}

\begin{frame}{Future work}

  \newcommand\futuresep{\vspace{3mm}}
  
  \begin{myitemize}
  \item We are getting close to the point where we can carry out likelihood-based inference for a flexible class of metapopulation models for measles.
Flexibility supports generation and testing of scientific hypotheses.
        
    \futuresep
    
  \item Measles was previously a motivating model system for POMP methods for single populations.

    \futuresep
    
    \item Many systems in ecology, epidemiology and elsewhere could be studied in a SpatPOMP framework.
    
\end{myitemize}

\end{frame}

\nocite{bjornstad01,grenfell04,breto19,rebeschini15,ng02,ionides21}

\begin{frame}[allowframebreaks]
\frametitle{References}
\bibliographystyle{apalike}
\bibliography{bib-sweden}
\end{frame}

%%% extra material

\begin{frame}
\frametitle{Filtering $U$-dimensional correlated Brownian motion}

\vspace{-3mm}

\begin{center}
\includegraphics[width=10cm]{bm_alt_plot-1.pdf}

\vspace{-1mm}

$\cov\big(X_{\unit,\time}-X_{\unit,\time-1},X_{\altUnit,\time}-X_{\altUnit,\time-1}\big) \sim 0.4^{|\unit-\altUnit|}_{}$

\end{center}

\end{frame}



\begin{frame}
\frametitle{Filtering $U$ units of Lorenz 96 toy atmospheric model} 

\vspace{-3mm}

\begin{center}
\includegraphics[width=10cm]{lz_loglik_plot-1.pdf}

\vspace{-1mm}

$dX_{\unit}(t) = \big \{  X_{\unit-1}(t) \big(X_{\unit+1}(t) - X_{\unit-2}(t)\big) - X_{\unit}(t) + F \big\} dt + \sigma \, dB_{\unit}(t)$

\end{center}

\end{frame}

\end{document}
