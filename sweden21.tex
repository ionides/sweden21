%\documentclass[handout]{beamer}
\documentclass{beamer}

\usepackage{amsmath,amsfonts,amssymb,graphicx} 
\input{header.tex}
\input{header-ms.tex}
\input{theorems1.tex}

\begin{document}

\begin{frame}
\begin{center}
  {\Large\bf Bagging and blocking: Inference via particle filters for interacting dynamic systems}
  
%% Infectious disease transmission is a nonlinear partially observed stochastic dynamic system with topical interest. For low-dimensional systems, models can be fitted to time series data using Monte Carlo particle filter methods. As dimension increases, for example when analyzing epidemics among multiple spatially coupled populations, basic particle filter methods rapidly degenerate. A collection of independent Monte Carlo calculations can be combined to give a global filtering solution with favorable theoretical scaling properties. The independent Monte Carlo calculations are called bootstrap replicates, and their aggregation is called a bagged filter. Bagged filtering is effective at likelihood evaluation for a model of measles transmission within and between cities. A blocked particle filter also works well at this task. Bagged and blocked particle filters can both be coerced into carrying out likelihood maximization by iterative application to an extension of the model that has stochastically perturbed parameters. Numerical results are carried out using the R package spatPomp.

\vspace{2mm}

Edward Ionides\\
University of Michigan, Department of Statistics

\vspace{8mm}

Statistics seminar series at \\
Chalmers University of Technology /
University of Gothenburg
\\
Tuesday November 23, 2021

%% 8:15am EST

\vspace{8mm}

Collaborators:\\
Kidus Asfaw, Ning Ning, Joonha Park and Aaron King\\

\end{center}

\end{frame}

\newcommand\challengeSep{\vspace{2mm}}


\begin{frame}
  \frametitle{Background}

  \myemph{The curse of dimensionality}. Particle filter (PF) methods are effective for inference on low-dimensional nonlinear partially observed stochastic dynamic systems. They scale exponentially badly.

  \vspace{10mm}
  
  \myemph{Bagged filters}. We study three algorithms that combine many non-interacting Monte Carlo processes.

  \vspace{3mm}
  
  \begin{myitemize}
  \item Unadapted bagged filter (UBF)
  \item adapted bagged filter (ABF)
    \item adapted bagged filter with intermediate resampling (ABF-IR)
  \end{myitemize}

  \vspace{10mm}

  \myemph{Metapopulation dynamics}. Bagged filters have theoretical and empirical scaling properties suited to collections of weakly coupled populations.

\end{frame}
\begin{frame}

  \frametitle{So, what about COVID-19?}

  \begin{myitemize}
  \item Researchers have developed many models for disease spread.
  \item Most of these build on the SIR (Susceptible-Infected-Removed) model that divides a population into three homogeneous classes.
    \item Extensions can include a latent period after infection, age structure, spatial structure, temperature, control policies.
    \item We may observe some fraction of cases.
    \item These models are partially observed stochastic dynamic systems.      
    \item Understanding of COVID-19 epidemiology draws on analysis of previous epidemics combined with assessment of limited available data.
      \item Methods that fit a general class of mechanistic models assist with formulating and testing scientific hypotheses.
      \end{myitemize}

\end{frame}


\begin{frame}

  \frametitle{What is a SpatPOMP?}

  \myemph{POMP} models are partially observed Markov processes, also known as state space models or hidden Markov models

  \vspace{6mm}
  
  \myemph{SpatPOMP} models are POMP models with a unit structure

    \vspace{6mm}
 
  \myemph{Latent Markov process}: $X_{\unit\comma\time}=X_{\unit}(t_\time)$, \hspace{1mm} $\unit\in 1{\mycolon}\Unit$, \hspace{1mm} $\time\in 1{\mycolon}\Time$

    \vspace{6mm}
 
  \myemph{Observation process}: $Y_{\unit\comma\time}$ depends only on $X_{\unit\comma\time}$

    \vspace{6mm}
 
The units could be a metapopulation, say cities in an epidemic model

\end{frame}


\begin{frame}
\frametitle{$U=40$ units for a coupled measles SEIR model}

\vspace{-5mm}

\begin{center}
\includegraphics[width=12cm]{slice_image_plot-1.pdf}


\end{center}

\vspace{-2mm}

Simulated data using a gravity model with geography, demography and transmission parameters corresponding to UK pre-vaccination measles.

%\end{center}

\end{frame}



\begin{frame}

  \vspace{8mm}
  
\noindent\begin{tabular}{l}
\hline
\inputSpace {\bf Bagged filter inputs, outputs and implicit loops.}\\
\hline
\inputSpace {\bf input:}
\\
simulator for $f_{\vec{X}_0}(\vec{x}_0)$ and $f_{\vec{X}_{\time}|\vec{X}_{\time-1}}(\vec{x}_{\time}\given \vec{x}_{\time-1})$\\
    evaluator for $f_{{Y}_{\unit\comma\time}|{X}_{\unit\comma\time}}({y}_{\unit\comma\time}\given {x}_{\unit\comma\time})$\\
    number of islands, $\Rep$\\
    neighborhood structure, $B_{\unit\comma\time}$\\
    data, $\data{\vec{y}}_{1:\Time}$\\
ABF and ABF-IR:    particles per island,  $\Np$\\
ABF-IR: number of intermediate timesteps, $\Ninter$ \\
ABF-IR: measurement variance parameterizations, ${\VtoTheta}_{\unit\comma\time}$ and ${\thetaToV}_{\unit\comma\time}$\\
ABF-IR: approximate process and observation mean functions, $\vec{\mu}$ and $h_{\unit\comma\time}$\\
\inputSpace {\bf output:}\\
  Log likelihood estimate, $\MC{\loglik}= \sum_{\time=1}^\Time\sum_{\unit=1}^\Unit \MC{\loglik}_{\unit\comma\time}$\\
\inputSpace {\bf implicit loops:}\\
$\; \unit \mbox{ in } \seq{1}{\Unit}$, 
$\; \time \mbox{ in } \seq{1}{\Time}$, 
$\; \rep \mbox{ in } \seq{1}{\Rep}$, 
$\; \np \mbox{ in } \seq{1}{\Np}$
%$\; \npgir \mbox{ in } \seq{1}{\Npgir}$
\lastLineSpace \\
\hline
\end{tabular}
\end{frame}


\begin{frame}
\renewcommand{\arraystretch}{1.8}.
\noindent\begin{tabular}{l}
\hline
\inputSpace {\bf {\TIF}. Unadapted bagged filter.}\\
\hline
%\inputSpace {\bf input:} From Table~1, with $\Np=1$ \\
%\hline
\firstLineSpace 
Simulate $\vec{X}^{\tif}_{0:\Time,\rep}\sim f_{\vec{X}_{0:\Time}}(\vec{x}_{0:\Time})$\\ 
Measurement weights,
$w^M_{\unit,\time,\rep}=f_{Y_{\unit,\time}|X_{\unit,\time}}(\data{y}_{\unit,\time}\given X^{\tif}_{\unit,\time,\rep})$
\\
Prediction weights, 
$w^P_{\unit,\time,\rep}=\prod_{(\tilde \unit,\tilde n)\in B_{\unit,\time}}
w^M_{\tilde\unit,\tilde n,\rep}$\\
$\MC{\loglik}_{\unit,\time}= 
\log\left(
  \sum_{\rep=1}^\Rep w^M_{\unit,\time,\rep}w^P_{\unit,\time,\rep}
\right)
-\log\left(
  \sum_{\rep=1}^\Rep w^P_{\unit,\time,\rep}
\right)
$
\rule[-5mm]{0mm}{8mm}
\\
\hline
\end{tabular}

\end{frame}


\begin{frame}

  \frametitle{The basic island filter is not as naive as it may first appear}

  \begin{myitemize}

  \item UBF seems naive. Particle filter (PF) method are well known to scale better with $\Time$ than unconditional simulations.

    \vspace{5mm}
    
  \item UBF scales well with $\Unit$ for weakly coupled systems.

    \vspace{5mm}

  \item With modern computers, large numbers of simulations are feasible even when $\Unit$ and $\Time$ are not small.

      \vspace{5mm}

    \item Initially we studied UBF as a theoretical toy. Then we found it is competitive in practice on some models of interest.
  \end{myitemize}

\end{frame}

\begin{frame}
  \frametitle{Adapted simulation: An easier problem than filtering}

  \begin{myitemize}
  \item
    We aim to make each island track the data in a weak sense that does not involve a solution to the full filtering problem.

\vspace{3mm}

  \item
    The adapted simulation problem is to draw from
$f_{\vec{X}_{\time}|\vec{Y}_{\time},\vec{X}_{\time-1}}
  \big(
    \vec{x}_{\time}\given \vec{\data{y}}_{\time},\vec{x}_{\time-1}
  \big)$.

\vspace{3mm}

  \item The adapted bagged filter (ABF) algorithm uses importance sampling to carry out adapted simulation on each island.


\vspace{3mm}

    \item ABF calculates the likelihood using the proper weight restricted to a neighborhood.
\end{myitemize}
    
\end{frame}



\begin{frame}

%\setlength\extrarowheight{5pt}
\renewcommand{\arraystretch}{1.2}.
\noindent\begin{tabular}{l}
\hline
{\bf 
{ABF}. Adapted bagged filter.}\inputSpace\\
\hline
%{\bf input:} From Table~1 \inputSpace \\
%\hline
\firstLineSpace
Initialize adapted simulation: $\vec{X}^{\IF}_{0,\rep} \sim f_{\vec{X}_0}(\vec{x}_0)$
\\
For $\time\ \mathrm{in}\ \seq{1}{\Time}$
\\
\asp  Proposals:
    $\vec{X}_{\time,\rep,\np}^{\IP} \sim 
    f_{\vec{X}_{\time}|\vec{X}_{\time-1}} 
    \big( \vec{x}_{\time}\given \vec{X}^{\IF}_{\time-1,\rep}\big)$
\\
\asp Measurement weights:
  $w^M_{\unit,\time,\rep,\np} = 
    f_{Y_{\unit,\time}|X_{\unit\comma\time}} 
    \big (\data{y}_{\unit\comma\time}\given X^{\IP}_{\unit\comma\time,\rep,\np}\big)$
\\
\asp  Adapted resampling weights:
  $w^{\IF}_{\time,\rep,\np} = 
    \prod_{\unit=1}^{\Unit} w^M_{\unit,\time,\rep,\np}$
\\
\asp
      Resampling:
        $\prob\big[\resampleIndex({\rep})=a \big] = w^{\IF}_{\time,\rep,a}
  \Big( 
  \sum_{\altNp=1}^{\Np} w^{\IF}_{\time,\rep,\altNp}
  \Big)^{-1}$
\\
\asp 
$\vec{X}^{\IF}_{\time,\rep} = \vec{X}^{\IP}_{\time,\rep,r(\rep)}$ 
\\
\asp % Prediction weights:
  $w^{\LCP}_{\unit,\time,\rep,\np}= \displaystyle
  \prod_{\altTime=1}^{\time-1}
  \Big[
    \frac{1}{\Np}\sum_{k=1}^{\Np}
    \hspace{1mm}
       \prod_{(\altUnit,\altTime)\in B^{[\altTime]}_{\unit,\time}} 
    \hspace{-1mm}
        w^M_{\altUnit,\altTime,\rep,k}
  \Big] \prod_{(\altUnit,\time)\in B^{[\time]}_{\unit,\time}} 
    \hspace{-1mm}
        w^M_{\altUnit,\time,\rep,\np}$
\\
End for
\\
$\displaystyle \MC{\loglik}_{\unit,\time}= 
\log\Bigg(
\frac{
\sum_{\rep=1}^\Rep \sum_{\np=1}^{\Np} w^M_{\unit,\time,\rep,\np}w^P_{\unit,\time,\rep,\np}
}{
\sum_{\rep=1}^\Rep \sum_{\np=1}^{\Np} w^P_{\unit,\time,\rep,\np}
}
\Bigg)
$
\rule[-8mm]{0mm}{10mm}
\\
\hline
\end{tabular}
\end{frame}


\begin{frame}
\frametitle{Intermediate resampling}



\begin{myitemize}
\item \myemph{Intermediate resampling} splits the time interval between observations into $\Ninter$ subintervals.

\vspace{2mm}

\item Reweighting and/or sampling at each subinterval uses a revised estimate of the anticipated measurement density at the end of the interval called a \myemph{guide function}.

\vspace{2mm}

\item This is applicable to continuous time models.

\vspace{2mm}

\item Intermediate resampling has useful theoretical and empirical properties \citep{delmoral15,park19}.

\vspace{2mm}

\item Intermediate resampling for adapted simulation within ABF gives the ABF-IR algorithm.

\vspace{2mm}

\item Intermediate resampling within PF gives the guided intermediate resampling filter (GIRF) of \citet{park19}, a generalization of the auxiliary particle filter of \citet{pitt99}.
  
\end{myitemize}

\end{frame}

\begin{frame}

  \resizebox{!}{45mm}{
\noindent\begin{tabular}{l}
\hline
{\bf {ABF-IR}. ABF  with  intermediate resampling.} 
\vspace{0.4mm} \\
\hline
\firstLineSpace
Initialize adapted simulation: $\vec{X}^{\IF}_{0,\rep} \sim f_{\vec{X}_0}(\vec{x}_0)$
\\
For $\time\ \mathrm{in}\ \seq{1}{\Time}$
\\
\asp Guide simulations:
    $\vec{X}_{\time,\rep,\npgir}^{G} \sim 
    f_{\vec{X}_{\time}|\vec{X}_{\time-1}} 
    \big( \vec{x}_{\time}\given \vec{X}^{\IF}_{\time-1,\rep} \big)$
\\
\asp Guide variance: $V_{\unit,\time,\rep}=
      \var \big\{
        h_{\unit\comma\time}\big( {X}_{\unit,\time,\rep,\npgir}^{G}\big), \npgir \mbox{ in } \seq{1}{\Npgir}
      \big\}$ 
\\
\asp $\guideFunc^{\resample}_{\time,0,\rep,\np}=1 \; \; $ and
$\; \vec{X}_{\time,0,\rep,\np}^{\GR}=\vec{X}^{\IF}_{\time-1,\rep}$
\\
\asp For $\ninter  \,\, \mathrm{in} \,\, \seq{1}{\Ninter}$
\\
\asp\asp Intermediate proposals:
        ${\vec{X}}_{\time,\ninter,\rep,\np}^{\GP}
          \sim {f}_{{\vec{X}}_{\time,\ninter}|{\vec{X}}_{\time,\ninter-1}}
          \big(\mydot|{\vec{X}}_{\time,\ninter-1,\rep,\np}^{\GR}\big)$ 
\\
\asp\asp 
        $\vec{\mu}^{\GP}_{\time,\ninter,\rep,\np} 
           = \vec{\mu}\big( \vec{X}^{\GP}_{\time,\ninter,\rep,\np},t_{\time,\ninter},t_{\time} \big)$
\\
\asp\asp      %Measurement variance at skeleton: 
        $V^{\mathrm{meas}}_{\unit,\time,\ninter,\rep,\np}
           = \thetaToV_{\unit}(\theta,\mu^{\GP}_{\unit,\time,\ninter,\rep,\np})$
%\\
           %\asp\asp  %Process variance:
           , \hspace{4mm}
        $V^{\mathrm{proc}}_{\unit,\time,\ninter,\rep}
          = V_{\unit,\time,\rep} \,
          \big(t_{\time}-t_{\time,\ninter}\big) \Big/
          \big(t_{\time}-t_{\time,0}\big)$ 
\\
\asp\asp
%      Moment matching:
        $\theta_{\unit,\time,\ninter,\rep,\np}= 
          \VtoTheta_{\unit}\big(
            V^{\mathrm{meas}}_{\unit,\time,\ninter,\rep,\np} + V^{\mathrm{proc}}_{\unit,\time,\ninter,\rep}, 
            \, \mu^{\GP}_{\unit,\time,\ninter,\rep,\np}
          \big)$
\\
\asp\asp  % Guide function: 
        $
\guideFunc_{\time,\ninter,\rep,\np}=
          \prod_{\unit=1}^{\Unit}
          f_{Y_{\unit,\time}|X_{\unit,\time}}
          \big(
            \data{y}_{\unit,\time}\given \mu^{\GP}_{\unit,\time,\ninter,\rep,\np} \giventh \theta_{\unit,\time,\ninter,\rep,\np} 
          \big)$
\\
\asp\asp Guide weights:
$w^G_{\time,\ninter,\rep,\np}= \guideFunc^{}_{\time,\ninter,\rep,\np}
         \big/ \guideFunc^{\resample}_{\time,\ninter-1,\rep,\np}$
\\
\asp\asp
      Resampling:
        $\prob\big[\resampleIndex({\rep,\np})=a \big] = w^G_{\time,\ninter,\rep,a}
\Big( \sum_{\altNp=1}^{\Np}w^G_{\time,\ninter,\rep,\altNp}\Big)^{-1}$
\\
\asp\asp
        $\vec{X}_{\time,\ninter,\rep,\np}^{\GR}=\vec{X}_{\time,\ninter,\rep,\resampleIndex({\rep,\np})}^{\GP}\; \; $ and
        $\; \guideFunc^{\resample}_{\time,\ninter,\rep,\np}= \guideFunc^{}_{\time,\ninter,\rep,\resampleIndex({\rep,\np})}\,$
\\
\asp
End For
\\
\asp
  Set $\vec{X}^{\IF}_{\time,\rep}=\vec{X}^{\GR}_{\time,\Ninter,\rep,1}$ 
\\ 
\asp Measurement weights:
  $w^M_{\unit,\time,\rep,\npgir} = 
    f_{Y_{\unit,\time}|X_{\unit,\time}} 
    \big (\data{y}_{\unit,\time}\given X^{G}_{\unit,\time,\rep,\npgir} \big)$
\\
\asp % Prediction weights:
  $w^{\LCP}_{\unit,\time,\rep,\npgir}= \displaystyle
  \prod_{\altTime=1}^{\time-1}
  \Big[
    \frac{1}{\Npgir}\sum_{a=1}^{\Npgir}
    \hspace{1mm}
       \prod_{(\altUnit,\altTime)\in B^{[\altTime]}_{\unit,\time}} 
    \hspace{-1mm}
        w^M_{\altUnit,\altTime,\rep,a}
  \Big] \prod_{(\altUnit,\time)\in B^{[\time]}_{\unit,\time}} 
    \hspace{-1mm}
        w^M_{\altUnit,\time,\rep,\npgir}$
\\
End for
\\
$\displaystyle \MC{\loglik}_{\unit,\time}= 
\log\Bigg(
\frac{
\sum_{\rep=1}^\Rep \sum_{\npgir=1}^{\Npgir} w^M_{\unit,\time,\rep,\npgir}w^P_{\unit,\time,\rep,\npgir}
}{
\sum_{\rep=1}^\Rep \sum_{\npgir=1}^{\Npgir} w^P_{\unit,\time,\rep,\npgir}
}
\Bigg)
$
\vspace{1mm}
\\
\hline
\end{tabular}
}
\end{frame}

\begin{frame}
  \frametitle{Software for SpatPOMP models}

  \begin{myitemize}
  \item We use the \code{asif}, \code{asifir} and \code{girf} implementations in the  R package \code{spatPomp} \citep{asfaw20github}.

\vspace{5mm}

  \item \code{spatPomp} offers a class `\code{spatPomp}' that extends the `\code{pomp}' class for POMP models in the R package \code{pomp} \citep{king16}.
%    \item All these algorithms are plug-and-play, 


\vspace{5mm}

  \item All methods available in \code{pomp} can formally be applied to `\code{spatPomp}' objects, though they may not be practically effective for spatiotemporal POMPs.
    \end{myitemize}
    
\end{frame}

\begin{frame}
\frametitle{Filtering $U$-dimensional correlated Brownian motion}

\vspace{-3mm}

\begin{center}
\includegraphics[width=10cm]{bm_alt_plot-1.pdf}

\vspace{-1mm}

$\cov\big(X_{\unit,\time}-X_{\unit,\time-1},X_{\altUnit,\time}-X_{\altUnit,\time-1}\big) \sim 0.4^{|\unit-\altUnit|}_{}$

\end{center}

\end{frame}

\begin{frame}
\frametitle{Filtering $U$ units of a coupled measles SEIR model}

\vspace{-3mm}

\begin{center}
\includegraphics[width=10cm]{mscale_loglik_plot-1.pdf}


\end{center}

\vspace{-2mm}

Simulated data using a gravity model with geography, demography and transmssion parameters corresponding to UK pre-vaccination measles.

%\end{center}

\end{frame}


\begin{frame}
\frametitle{Filtering $U$ units of Lorenz 96 toy atmospheric model} 

\vspace{-3mm}

\begin{center}
\includegraphics[width=10cm]{lz4_loglik_plot-1.pdf}

\vspace{-1mm}

$dX_{\unit}(t) = \big \{  X_{\unit-1}(t) \big(X_{\unit+1}(t) - X_{\unit-2}(t)\big) - X_{\unit}(t) + F \big\} dt + \sigma \, dB_{\unit}(t)$

\end{center}

\end{frame}

\begin{frame}
\frametitle{From filtering to parameter inference}

\begin{myitemize}
\item Log likelihood evaluation in principle enables likelihood-based or Bayesian inference.

\vspace{3mm}

\item Iterated filtering maximizes the likelihood for PF or GIRF \citep{ionides15}.

\vspace{3mm}

\item Particle Markov chain Monte Carlo can be applied with any likelihood estimate \citep{andrieu10}. It is numerically intractable when Monte Carlo estimates are costly and noisy.

\vspace{3mm}

\item Extending iterated filtering to island filters is future work.

\end{myitemize}

\end{frame}

\begin{frame}

\frametitle{Measles likelihood slices for $G$ and $\mu_{IR}$ via ABF}

\vspace{-3mm}

%\hspace{-5mm}
\includegraphics[width=5.5cm]{slice_plot-1.pdf}
\includegraphics[width=5.5cm]{gammaSlice_plot-1.pdf}

\begin{myitemize}
\item 
Simulating $15$ year of data from $U=40$ cities for the measles model.
\item
The gravitational coupling constant $G$ is fairly weakly identified: a week of computing on a 30 core machine gives Monte Carlo error on the same scale as the statistical uncertainty.
\item
The recovery rate $\mu_{IR}$ is well identified.
\end{myitemize}
\vspace{-2mm}

\end{frame}

\begin{frame}
\begin{theorem}  \label{thm:tif}
Let $\MC{\loglik}$ denote the Monte Carlo likelihood approximation constructed by UBF, ABF or ABF-IR.
Consider a limit with a growing number of islands, $\Rep\to\infty$.
Suppose regularity assumptions listed in the arXiv preprint.
There are quantities $\el(\Unit,\Time) = O(1)$ and $V(\Unit,\Time)=O(\Unit^2\Time^2)$ such that
\begin{equation}
%\label{th1:lik:bound}
\nonumber
\Rep^{1/2}\big[ \MC{\loglik}-\loglik-\el\Unit\Time \big]  \xrightarrow[\Rep \rightarrow \infty]{d} \normal\big[0,V\big],
\end{equation}
where $\xrightarrow[\Rep \rightarrow \infty]{d}$ denotes convergence in distribution and $\normal[\mu,\Sigma]$ is the normal distribution with mean $\mu$ and variance $\Sigma$.
If an additional spatiotemporal mixing assumption holds, we obtain an improved variance bound
\begin{equation}
%\label{th1:lik:bound2}
\nonumber
%V < \ThmOneVarBound.
V(\Unit,\Time) = O(\Unit\Time)
\end{equation}
\end{theorem}
\end{frame}




\begin{frame}[allowframebreaks]
\frametitle{References}
\bibliographystyle{apalike}
\bibliography{bib-sweden}
\end{frame}

\end{document}
